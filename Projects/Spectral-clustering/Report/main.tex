\documentclass[twoside,11pt]{article}

% Any additional packages needed should be included after jmlr2e.
% Note that jmlr2e.sty includes epsfig, amssymb, natbib and graphicx,
% and defines many common macros, such as 'proof' and 'example'.
%
% It also sets the bibliographystyle to plainnat; for more information on
% natbib citation styles, see the natbib documentation, a copy of which
% is archived at http://www.jmlr.org/format/natbib.pdf

\usepackage{jmlr2e}

% Definitions of handy macros can go here

\newcommand{\dataset}{{\cal D}}
\newcommand{\fracpartial}[2]{\frac{\partial #1}{\partial  #2}}

% Heading arguments are {volume}{year}{pages}{submitted}{published}{author-full-names}

\jmlrheading{19}{2018}{1-8}{12/00}{13/00}{Jia Guo, Masaya Tsukamoto, Zihan Wang and Guangting Yu}

% Short headings should be running head and authors last names

\ShortHeadings{Spectral Clustering}{Guo, Tsukamoto, Wang and Yu}
\firstpageno{1}

\begin{document}

\title{Spectral Clustering}

\author{\name Jia Guo \email guojia@umich.edu \\
        \addr Department of Mathematics\\
        University of Michigan\\
        Ann Arbor, MI 48109, USA
        \AND
        \name Masaya Tsukamoto \email masayats@umich.edu \\
        \addr Department of Mathematics\\
        University of Michigan\\
        Ann Arbor, MI 48109, USA
        \AND
        \name Zihan Wang \email wzihan@umich.edu \\
        \addr Department of Climate and Space Sciences and Engineering\\
        University of Michigan\\
        Ann Arbor, MI 48109, USA
        \AND
        \name Guangting Yu \email yugtmath@umich.edu \\
        \addr Department of Mathematics\\
        University of Michigan\\
        Ann Arbor, MI 48109, USA}

\editor{Nobody}

\maketitle

\begin{abstract}
Nothing
\end{abstract}

\begin{keywords}
Spectral Clustering
\end{keywords}

\section{Introduction}


% Acknowledgements should go at the end, before appendices and references

\acks{None}

% Manual newpage inserted to improve layout of sample file - not
% needed in general before appendices/bibliography.

\newpage

\appendix
\section*{Appendix A.}
\label{app:theorem}

% Note: in this sample, the section number is hard-coded in. Following
% proper LaTeX conventions, it should properly be coded as a reference:

%In this appendix we prove the following theorem from
%Section~\ref{sec:textree-generalization}:

In this appendix we prove the following theorem from
Section~6.2:


\begin{theorem}
Let $u,v,w$ be discrete variables such that $v, w$ do
not co-occur with $u$ (i.e., $u\neq0\;\Rightarrow \;v=w=0$ in a given
dataset $\dataset$). Let $N_{v0},N_{w0}$ be the number of data points for
which $v=0, w=0$ respectively, and let $I_{uv},I_{uw}$ be the
respective empirical mutual information values based on the sample
$\dataset$. Then
\[ N_{v0} \;>\; N_{w0}\;\;\Rightarrow\;\;I_{uv} \;\leq\;I_{uw} \]
with equality only if $u$ is identically 0.
\end{theorem}






\vskip 0.2in
\bibliography{sample}

\end{document}